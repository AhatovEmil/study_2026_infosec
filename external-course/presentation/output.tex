% Options for packages loaded elsewhere
\PassOptionsToPackage{unicode}{hyperref}
\PassOptionsToPackage{hyphens}{url}
\documentclass[
  ignorenonframetext,
  aspectratio=169,
  russian,
]{beamer}
\newif\ifbibliography
\usepackage{pgfpages}
\setbeamertemplate{caption}[numbered]
\setbeamertemplate{caption label separator}{: }
\setbeamercolor{caption name}{fg=normal text.fg}
\beamertemplatenavigationsymbolsempty
% remove section numbering
\setbeamertemplate{part page}{
  \centering
  \begin{beamercolorbox}[sep=16pt,center]{part title}
    \usebeamerfont{part title}\insertpart\par
  \end{beamercolorbox}
}
\setbeamertemplate{section page}{
  \centering
  \begin{beamercolorbox}[sep=12pt,center]{section title}
    \usebeamerfont{section title}\insertsection\par
  \end{beamercolorbox}
}
\setbeamertemplate{subsection page}{
  \centering
  \begin{beamercolorbox}[sep=8pt,center]{subsection title}
    \usebeamerfont{subsection title}\insertsubsection\par
  \end{beamercolorbox}
}
% Prevent slide breaks in the middle of a paragraph
\widowpenalties 1 10000
\raggedbottom
\AtBeginPart{
  \frame{\partpage}
}
\AtBeginSection{
  \ifbibliography
  \else
    \frame{\sectionpage}
  \fi
}
\AtBeginSubsection{
  \frame{\subsectionpage}
}
\usepackage{iftex}
\ifPDFTeX
  \usepackage[T1]{fontenc}
  \usepackage[utf8]{inputenc}
  \usepackage{textcomp} % provide euro and other symbols
\else % if luatex or xetex
  \usepackage{unicode-math} % this also loads fontspec
  \defaultfontfeatures{Scale=MatchLowercase}
  \defaultfontfeatures[\rmfamily]{Ligatures=TeX,Scale=1}
\fi
\usepackage{lmodern}
\usetheme[]{metropolis}
\usefonttheme{serif} % use mainfont rather than sansfont for slide text
\ifPDFTeX\else
  % xetex/luatex font selection
  \setmainfont[Ligatures=TeX]{PT Serif}
  \setsansfont[Ligatures=TeX,Scale=MatchLowercase]{PT Sans}
  \setmonofont[Scale=MatchLowercase,Scale=0.9]{PT Mono}
\fi
% Use upquote if available, for straight quotes in verbatim environments
\IfFileExists{upquote.sty}{\usepackage{upquote}}{}
\IfFileExists{microtype.sty}{% use microtype if available
  \usepackage[]{microtype}
  \UseMicrotypeSet[protrusion]{basicmath} % disable protrusion for tt fonts
}{}
\makeatletter
\@ifundefined{KOMAClassName}{% if non-KOMA class
  \IfFileExists{parskip.sty}{%
    \usepackage{parskip}
  }{% else
    \setlength{\parindent}{0pt}
    \setlength{\parskip}{6pt plus 2pt minus 1pt}}
}{% if KOMA class
  \KOMAoptions{parskip=half}}
\makeatother
\usepackage{graphicx}
\makeatletter
\newsavebox\pandoc@box
\newcommand*\pandocbounded[1]{% scales image to fit in text height/width
  \sbox\pandoc@box{#1}%
  \Gscale@div\@tempa{\textheight}{\dimexpr\ht\pandoc@box+\dp\pandoc@box\relax}%
  \Gscale@div\@tempb{\linewidth}{\wd\pandoc@box}%
  \ifdim\@tempb\p@<\@tempa\p@\let\@tempa\@tempb\fi% select the smaller of both
  \ifdim\@tempa\p@<\p@\scalebox{\@tempa}{\usebox\pandoc@box}%
  \else\usebox{\pandoc@box}%
  \fi%
}
% Set default figure placement to htbp
\def\fps@figure{htbp}
\makeatother
\ifLuaTeX
\usepackage[bidi=basic,shorthands=off,provide=*]{babel}
\else
\usepackage[bidi=default,shorthands=off,provide=*]{babel}
\fi
\ifPDFTeX
\else
\babelfont{rm}[Ligatures=TeX]{PT Serif}
\fi
\ifLuaTeX
  \usepackage{selnolig} % disable illegal ligatures
\fi
\setlength{\emergencystretch}{3em} % prevent overfull lines
\providecommand{\tightlist}{%
  \setlength{\itemsep}{0pt}\setlength{\parskip}{0pt}}
\metroset{progressbar=frametitle,sectionpage=progressbar,numbering=fraction}
\makeatletter
\beamer@ignorenonframefalse
\makeatother
\usepackage{bookmark}
\IfFileExists{xurl.sty}{\usepackage{xurl}}{} % add URL line breaks if available
\urlstyle{same}
\hypersetup{
  pdftitle={Презентация по лабораторной работе №1},
  pdfauthor={Ахатов Э.Э.},
  pdflang={ru-RU},
  hidelinks,
  pdfcreator={LaTeX via pandoc}}

\title{\texorpdfstring{Презентация по лабораторной работе
№1}{Презентация по лабораторной работе №1}}
\subtitle{\texorpdfstring{Основы информационной
безопасности}{Основы информационной безопасности}}
\author{\texorpdfstring{Ахатов Э.Э.}{Ахатов Э.Э.}}
\date{16 февраля 2026}
\institute{Российский университет дружбы народов, Москва, Россия}

\begin{document}
\frame{\titlepage}

\section{Информация}\label{ux438ux43dux444ux43eux440ux43cux430ux446ux438ux44f}

\begin{frame}{Докладчик}
\protect\phantomsection\label{ux434ux43eux43aux43bux430ux434ux447ux438ux43a}
\begin{columns}[c]
\end{columns}

\begin{itemize}
\tightlist
\item
  Ахатов Эмиль Эрнстович
\item
  студентка группы НКАбд-04-24
\item
  Российский университет дружбы народов
\end{itemize}

::: :::

::: ::::::::::::::
\end{frame}

\begin{frame}{Цель}
\protect\phantomsection\label{ux446ux435ux43bux44c}
Целью данной работы является приобретение практических навыков установки
операционной системы на виртуальную машину, настройки ми- нимально
необходимых для дальнейшей работы сервисов.
\end{frame}

\begin{frame}{Задание}
\protect\phantomsection\label{ux437ux430ux434ux430ux43dux438ux435}
\begin{enumerate}
\tightlist
\item
  Установка и настройка операционной системы.
\item
  Найти следующую информацию:

  \begin{enumerate}
  \tightlist
  \item
    Версия ядра Linux (Linux version).
  \item
    Частота процессора (Detected Mhz processor).
  \item
    Модель процессора (CPU0).
  \item
    Объем доступной оперативной памяти (Memory available).
  \item
    Тип обнаруженного гипервизора (Hypervisor detected).
  \item
    Тип файловой системы корневого раздела.
  \end{enumerate}
\end{enumerate}
\end{frame}

\begin{frame}{Выполнение лабораторной работы}
\protect\phantomsection\label{ux432ux44bux43fux43eux43bux43dux435ux43dux438ux435-ux43bux430ux431ux43eux440ux430ux442ux43eux440ux43dux43eux439-ux440ux430ux431ux43eux442ux44b}
Я создал новую виртуальную машину, выделил 8гб оперативной памяти, 4
ядра процессора и максимальное колличество видеопамяти,40 гб дискового
пространства,выбрал окружение средства разработки в дополнительном
програмном обеспечении, отключил kdump.Так как этот процесс полностью
повторяется с прошлым курсом,я не видел необходимости скриншотить каждый
шаг.
\end{frame}

\section{Выполнение дополнительного
задания}\label{ux432ux44bux43fux43eux43bux43dux435ux43dux438ux435-ux434ux43eux43fux43eux43bux43dux438ux442ux435ux43bux44cux43dux43eux433ux43e-ux437ux430ux434ux430ux43dux438ux44f}

\begin{frame}{1}
\protect\phantomsection\label{section}
\begin{figure}
\centering
\pandocbounded{\includegraphics[keepaspectratio,alt={окно терминала}]{image/1.png}}
\caption{окно терминала}
\end{figure}
\end{frame}

\begin{frame}{2}
\protect\phantomsection\label{section-1}
\begin{figure}
\centering
\pandocbounded{\includegraphics[keepaspectratio,alt={окно терминала}]{image/2.png}}
\caption{окно терминала}
\end{figure}
\end{frame}

\begin{frame}{Вывод}
\protect\phantomsection\label{ux432ux44bux432ux43eux434}
Я приобрел практические навыки установки операционной системы на
виртуальную машину, настройки ми- нимально необходимых для дальнейшей
работы сервисов.

:::
\end{frame}

\end{document}
